\subsection{Helper methods}
There are 4 helper methods to handle our abstract map:\\

$\begin{array}{rl}
\wsf{contains} & \in \AMap \times \AKey \rightarrow \wbf{Boolean} \\
\wsf{lookup} & \in \AMap \times \AKey \rightarrow \AVal \times \wbf{Boolean} \\
\wsf{update} & \in \AMap \times \AKey \times \AVal \rightarrow \AMap \\
\wsf{delete} & \in \AMap \times \AKey \rightarrow \AMap \\
\end{array}$\\\\

\textbf{Monotonicity } ??\\

\textbf{Soundness }
We proved the soundness of those methods at section \ref{sec:soundness}.\\

\textbf{Termination }
We call a abstract map \emph{infinite} if $\Dom$ of the abstract map is infinite,
and \emph{finite} otherwise.
Even though the definition of $\AMap$ does not guarantee a abstract map to be finite,
the inputs of the helper methods will never be infinite during the analysis.
Because every abstract map created by alpha function during the analysis is finite
as the program code is finite, and if the input abstract map is finite, 
the resulted abstract map of helper methods is also finite by the definition of the methods.
So we assume that every abstract map used as inputs of helper method are finite.
Thus the helper methods always terminates.\\

\subsubsection{contains}
$\wsf{contains}$ is used to check whether the given abstract key
has a mapping in given abstract map.\\

$\begin{array}{l}
\wsf{contains} \in \AMap \times \AKey \rightarrow \wbf{Boolean} \\\\
\wsf{contains}(\bot_m,\ \hat{k}) = \bot_b \\
\wsf{contains}(\hat{m},\ \bot_k) = \bot_b \\
\wsf{contains}(\hat{m},\ \hat{k}) = \hat{b} \\
\quad \begin{array}{ll} \textrm{where }
& \hat{b} = \left \{ \begin{array}{ll}
\top_b & \textrm{if } \emph{domIn?} \land \gamma_k(\hat{k}) \not\subseteq \Defset(\hat{m}) \\
\hat{\texttt{true}} & \textrm{if } \emph{domIn?} \land \gamma_k(\hat{k}) \subseteq \Defset(\hat{m}) \\
\hat{\texttt{false}} & \textrm{if } \neg \emph{domIn?} \\
\end{array} \right.\\
& \emph{domIn?} = \exists \hat{k}' \in \Dom(\hat{m}) \cdot
\wsf{isRelated}(\hat{k},\ \hat{k}')\\
\end{array}\\
\end{array}$

\subsubsection{lookup}
$\wsf{lookup}$ gets the abstract map and abstract key,
and returns the abstract value mapped to the keys related to given abstract key.
The abstract boolean value returned with the abstract value indicates
whether the returned value is definite or not.\\

$\begin{array}{l}
\wsf{lookup} \in \AMap \times \AKey \rightarrow \AVal \times \wbf{Boolean} \\\\

\wsf{lookup}(\bot_m,\ \hat{k}) = (\bot_v,\ \bot_b) \\
\wsf{lookup}(\hat{m},\ \bot_{k}) = (\bot_v,\ \bot_b) \\

\wsf{lookup}(\hat{m},\ \hat{k}) = (\emph{local},\ \hat{b}) \\
\quad \begin{array}{rl} \textrm{where}
& \emph{local} = \bigsqcup_v \left \{ 
\Map(\hat{m})(\hat{k}') \mid \hat{k}' \in \emph{S} \right \} \vspace{1mm}\\
& \hat{b} = \left \{ \begin{array}{ll}
\top_b & \textrm{if } \emph{S} \neq \varnothing \land 
\gamma_k(\hat{k}) \not\subseteq \Defset(\hat{m}) \\
\hat{\texttt{true}} & \textrm{if } \emph{S} \neq \varnothing \land 
\gamma_k(\hat{k}) \subseteq \Defset(\hat{m}) \\
\hat{\texttt{false}} & \textrm{if } \emph{S} = \varnothing \\
\end{array} \right. \vspace{1mm} \\
& \emph{S} = \{ \hat{k}' \mid \hat{k}' \in \Dom(\hat{m})
\land \wsf{isRelated}(\hat{k},\ \hat{k}') \}\\
\end{array} \\
\end{array}$

\subsubsection{update}
$\wsf{update}$ calculate updated abstract map of given map with given key and value.
If the given abstract key is \emph{exact}, ignore existing mapped value to the key.
Otherwise, join the existing mapped value with the new value.
It is important to determine whether the given key is \emph{exact} or not,
in order to the result of the $\wsf{update}$ methods be precise.\\

$\begin{array}{l}
\wsf{update} \in \AMap \times \AKey \times \AVal \rightarrow \AMap \\\\

\wsf{update}(\bot_m,\ \hat{k},\ \hat{v}) = \bot_m \\
\wsf{update}(\hat{m},\ \bot_k,\ \hat{v}) = \bot_m \\
\wsf{update}(\hat{m},\ \hat{k},\ \bot_{v}) = \hat{m} \\

\wsf{update}(\hat{m},\ \hat{k},\ \hat{v}) = \langle \emph{map},\ \emph{defset} \rangle \vspace{1mm}\\
\quad \begin{array}{rl} \textrm{where}
& \emph{map} = \left \{ \begin{array}{ll}
\Map(\hat{m}) \ast [ \hat{k} \mapsto \hat{v} ]
& \textrm{if } \hat{k} \not\in \Dom(\hat{m}) \\

(\Map(\hat{m}) - \hat{k}) \ast [ \hat{k} \mapsto \hat{v}]
& \textrm{if } \emph{exact?} \land \hat{k}\in \Dom(\hat{m}) \\

(\Map(\hat{m}) - \hat{k})
\ast [ \hat{k} \mapsto \hat{v} \sqcup_v \Map(\hat{m})(\hat{k}) ]
& \textrm{if } \neg \emph{exact?} \land \hat{k}\in \Dom(\hat{m}) \\
\end{array} \right. \vspace{1mm}\\

& \emph{defset} = \left \{ \begin{array}{ll}
\Defset(\hat{m}) \cup \gamma_k(\hat{k}) & \textrm{if} ~ \emph{exact?} \land \hat{k}\in \Dom(\hat{m}) \\
\Defset(\hat{m}) & \textrm{otherwise}\\
\end{array} \right. \vspace{1mm}\\

& \emph{exact?} = \mid \gamma_k(\hat{k}) \mid = 1 \\
\end{array} \\
\end{array} $

\subsubsection{delete}
For given abstract map and abstract key, 
$\wsf{delete}$ returns the new abstract map 
which have no value mapped to the given key.
Similar to $\wsf{update}$ method, 
if the given key is not \emph{exact},
$\wsf{delete}$ cannot change the input map.
Thus it is important to determine \emph{exact}ness of the given key.\\

$\begin{array}{l}
\wsf{delete} \in \AMap \times \AKey \rightarrow \AMap \\\\

\wsf{delete}(\bot_m,\ \hat{k}) = \bot_m \\
\wsf{delete}(\hat{m},\ \bot_k) = \bot_m \\

\wsf{delete}(\hat{m},\ \hat{k}) =
\langle \emph{map},\ \Defset(\hat{m}) \setminus \gamma_k(\hat{k}) \rangle \vspace{1mm}\\
\quad \begin{array}{rl} \textrm{where}
& \emph{map} = \left \{ \begin{array}{ll}
(\Map(\hat{m}) - \hat{k})
& \textrm{if } \emph{exact?} \land \hat{k} \in \Dom(\hat{m}) \\
\Map(\hat{m})
& \textrm{otherwise} \\
\end{array} \right. \vspace{1mm}\\

& \emph{exact?} = \mid \gamma_k(\hat{k}) \mid = 1 \\
\end{array} \\
\end{array}$

\subsection{Soundness} \label{sec:soundness}
\newtheorem{thm}{Theorem}
\begin{thm} \normalfont
(\textit{Soundness of} $\wsf{contains}$)
$\forall m_1 \in \CMap, k_1 \in \CKey :$\\
If $\exists \hat{m}_2 \in \AMap \cdot m_1 \in \gamma_m(\hat{m}_2)$
and $\exists \hat{k}_2 \in \AKey \cdot k_1 \in \gamma_k(\hat{k}_2)$,
then $\textsf{contains}(m_1,\ k_1) \in \gamma_b(\wsf{contains}(\hat{m}_2,\ \hat{k}_2))$.
\end{thm}
\textbf{Proof } $\forall m_1 \in \CMap, k_1 \in \CKey$, 
let $\hat{m}_2 \in \AMap \cdot m_1 \in \gamma_m(\hat{m}_2)$ 
and $\hat{k}_2 \in \AKey \cdot k_1 \in \gamma_k(\hat{k}_2)$.\\
Let $\hat{m}_1 = \alpha_m(\{ m_1 \})$ and $\hat{k}_1 = \alpha_k(\{ k_1 \})$,
then $\hat{m}_1 \po_m \hat{m}_2$ and $\hat{k}_1 \po_k \hat{k}_2$.
\begin{itemize}
\item If $\textsf{contains}(m_1, k_1) = \texttt{true}$, then $k_1 \in \Dom(m_1)$.\\
$\Dom(\hat{m}_1)  = \{ \alpha_k(\{k\}) \mid k \in \Dom(m_1) \}$
by definition of $\alpha_m$, thus $\hat{k}_1 \in \Dom(\hat{m}_1)$.\\
Also, $\exists \hat{k} \in \Dom(\hat{m}_2) 
\cdot \hat{k}_1 \po_k \hat{k} \land \Map(\hat{m}_1)(\hat{k}_1) \po_v \Map(\hat{m}_2)(\hat{k})$,
by definition of $\po_m$.\\
$k_1 \in \gamma_k(\hat{k}_1) \subseteq \gamma_k(\hat{k})$ by monotonicity of $\gamma_k$.\\ 
$\wsf{isRelated}(\hat{k}_2, \hat{k})$ must be \texttt{true},
since $k_1 \in \gamma_k(\hat{k}) \land k_1 \in \gamma_k(\hat{k}_2)$. \\
Therefore $\wsf{contains}(\hat{m}_2, \hat{k}_2)$ should be either $\top_b$ or $\hat{\texttt{true}}$.\\
Thus $\textsf{contains}(m_1, k_1) \in \gamma_b(\wsf{contains}(\hat{m}_2, \hat{k}_2))$.
\item If $\textsf{contains}(m_1, k_1) = \texttt{false}$, then $k_1 \not\in \Dom(m_1)$.\\
$\Defset(\hat{m}_2) \subseteq \Defset(\hat{m}_1)$ by definition of $\po_m$,
and $\Defset(\hat{m}_1) = \Dom(m_1)$ by definition of $\alpha_m$.\\
$k_1 \not\in \Defset(\hat{m}_2)$ because $k_1 \not\in \Dom(m_1) = \Defset(\hat{m}_1)$. \\
$\gamma_k(\hat{k}_2) \not\subseteq \Defset(\hat{m}_2)$, since
$k_1 \not\in \Defset(\hat{m}_2)$ but $k_1 \in \gamma_k(\hat{k}_2)$.\\
Therefore $\wsf{contains}(\hat{m}_2, \hat{k}_2)$ should be either $\top_b$ or $\hat{\texttt{false}}$.\\
Thus $\textsf{contains}(m_1, k_1) \in \gamma_b(\wsf{contains}(\hat{m}_2, \hat{k}_2))$.
\end{itemize}


\begin{thm} \normalfont
(\textit{Soundness of} $\wsf{lookup}$)
$\forall m_1 \in \CMap, k_1 \in \CKey :$\\
If $\exists \hat{m}_2 \in \AMap \cdot m_1 \in \gamma_m(\hat{m}_2)$,
$\exists \hat{k}_2 \in \AKey \cdot k_1 \in \gamma_k(\hat{k}_2)$,
and $(\hat{v}, \hat{b}) = \wsf{lookup}(\hat{m}_2,\ \hat{k}_2)$ 
for $\hat{v} \in \AVal, \hat{b} \in \wbf{Boolean}$,
then $\textsf{lookup}(m_1,\ k_1) \in \gamma_v(\hat{v}) \cup \gamma_b(\hat{b})$.
\end{thm}
\textbf{Proof } $\forall m_1 \in \CMap, k_1 \in \CKey$, 
let $\hat{m}_2 \in \AMap \cdot m_1 \in \gamma_m(\hat{m}_2)$,
$\hat{k}_2 \in \AKey \cdot k_1 \in \gamma_k(\hat{k}_2)$,
and $(\hat{v},\ \hat{b}) = \wsf{lookup}(\hat{m}_2,\ \hat{k}_2)$.\\
Let $\hat{m}_1 = \alpha_m(\{ m_1 \})$ and $\hat{k}_1 = \alpha_k(\{ k_1 \})$,
then $\hat{m}_1 \po_m \hat{m}_2$ and $\hat{k}_1 \po_k \hat{k}_2$.
\begin{itemize}
\item If $\textsf{lookup}(m_1, k_1) = m_1(k_1) \in \CVal$, then $k_1 \in \Dom(m_1)$.\\
$\hat{v} = \bigsqcup_v \emph{V}$
where $\emph{V} = \{ \Map(\hat{m}_2)(\hat{k}) \mid \hat{k} \in \emph{S} \}$
and $\emph{S} = \{ \hat{k} \mid \hat{k} \in \Dom(\hat{m}_2) \land \wsf{isRelated}(\hat{k}, \hat{k}_2) \}$
by definition of $\wsf{lookup}$. \vspace{1mm} \\
$\Dom(\hat{m}_1)  = \{ \alpha_k(\{k\}) \mid k \in \Dom(m_1) \}$
by definition of $\alpha_m$, thus $\hat{k}_1 \in \Dom(\hat{m}_1)$.\\
Also, $\exists \hat{k} \in \Dom(\hat{m}_2) 
\cdot \hat{k}_1 \po_k \hat{k} \land \Map(\hat{m}_1)(\hat{k}_1) \po_v \Map(\hat{m}_2)(\hat{k})$,
by definition of $\po_m$.\\
$k_1 \in \gamma_k(\hat{k}_1) \subseteq \gamma_k(\hat{k})$ by monotonicity of $\gamma_k$.\\ 
$\wsf{isRelated}(\hat{k}_2, \hat{k})$ must be \texttt{true},
since $k_1 \in \gamma_k(\hat{k}) \land k_1 \in \gamma_k(\hat{k}_2)$,
thus $\hat{k} \in \emph{S} $.\\
It means $\Map(\hat{m}_2)(\hat{k}) \in \emph{V}$,
so that $\Map(\hat{m}_2)(\hat{k}) \po_v \hat{v}$. \vspace{1mm} \\
$m_1(k_1) \in \gamma_v (\Map(\hat{m}_1)(\hat{k}_1))$ by definition of $\alpha_m$.\\
$m_1(k_1) \in \gamma_v (\Map(\hat{m}_1)(\hat{k}_1))
\subseteq \gamma_v (\Map(\hat{m}_2)(\hat{k})) \subseteq \gamma_v (\hat{v})$,
by monotonicity of $\gamma_v$.\\
Therefore, $\textsf{lookup}(m_1, k_1) \in \gamma_v(\hat{v}) \cup \gamma_b(\hat{b})$.
\item If $\textsf{lookup}(m_1, k_1) = \texttt{false}$, then $k_1 \not\in \Dom(m_1)$.\\
$\Defset(\hat{m}_2) \subseteq \Defset(\hat{m}_1)$ by definition of $\po_m$,
and $\Defset(\hat{m}_1) = \Dom(m_1)$ by definition of $\alpha_m$.\\
$k_1 \not\in \Defset(\hat{m}_2)$ because $k_1 \not\in \Dom(m_1) = \Defset(\hat{m}_1)$. \\
$\gamma_k(\hat{k}_2) \not\subseteq \Defset(\hat{m}_2)$, since
$k_1 \not\in \Defset(\hat{m}_2)$ but $k_1 \in \gamma_k(\hat{k}_2)$.\\
Then $\hat{b}$ should be either $\top_b$ or $\hat{\texttt{false}}$,
so that $\texttt{false} \in \gamma_b(\hat{b})$. \\
Thus $\textsf{lookup}(m_1, k_1) \in \gamma_v(\hat{v}) \cup \gamma_b(\hat{b})$.
\end{itemize}


\begin{thm} \normalfont
(\textit{Soundness of} $\wsf{update}$) 
$\forall m_1 \in \CMap, k_1 \in \CKey, v_1 \in \CVal :$\\
If $\exists \hat{m}_2 \in \AMap \cdot m_1 \in \gamma_m(\hat{m}_2)$,
$\exists \hat{k}_2 \in \AKey \cdot k_1 \in \gamma_k(\hat{k}_2)$,
and $\exists \hat{v}_2 \in \AVal \cdot v_1 \in \gamma_v(\hat{v}_2)$,
then $\textsf{update}(m_1,\ k_1,\ v_1) \in \gamma_m(\wsf{update}(\hat{m}_2,\ \hat{k}_2,\ \hat{v}_2))$.
\end{thm}


\begin{thm} \normalfont
(\textit{Soundness of} $\wsf{delete}$) 
$\forall m \in \CMap, k \in \CKey :$\\
If $\exists \hat{m} \in \AMap \cdot m \in \gamma_m(\hat{m})$
and $\exists \hat{k} \in \AKey \cdot k \in \gamma_k(\hat{k})$,
then $\textsf{delete}(m,\ k) \in \gamma_m(\wsf{delete}(\hat{m},\ \hat{k}))$.
\end{thm}
