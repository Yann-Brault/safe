\chapter{Overview}\label{c:1:overview}

\section{Introduction to \safe 1.0}
\label{s:2:1:safe1}
Analyzing real-world JavaScript web applications is a challenging task.
On top of understanding the semantics of JavaScript~\cite{ecma5},
it requires modeling of web documents~\cite{W3C}, platform objects~\cite{tizen},
and interactions between them.
Not only JavaScript itself but also its usage patterns are extremely dynamic~\cite{dynamic,eval}.
Most of web applications load JavaScript code dynamically,
which makes pure static analysis approaches inapplicable.

To analyze JavaScript web applications in the wild mostly statically,
we have developed \safe and extended it with various approaches.
We first described quirky language features and semantics of JavaScript
that make static analysis difficult and designed \safe to analyze pure JavaScript
benchmarks~\cite{fool12}.  It provides a default static analyzer based on
the abstract interpretation framework~\cite{ai77},
and it supports flow-sensitive and context-sensitive analyses of stand-alone JavaScript programs.
It performs several preprocessing steps on JavaScript code
to address some quirky semantics of JavaScript such as
the \texttt{with} statement~\cite{dls13}.
The pluggable and scalable design of the framework allowed experiments with JavaScript variants
like adding a module system~\cite{oopsla12,modularity14}
and detecting code clones~\cite{emse16}.


We then extended \safe
to model web application execution environments of various browsers~\cite{ase15a} and
platform-specific library functions~\cite{fse14,safets}.  To provide a faithful (partial) model of browsers,
we support the configurability of HTML/DOM tree abstraction levels 
so that users can adjust a trade-off between analysis performance and precision 
depending on their applications.  To analyze interactions between applications
and platform-specific libraries specified in Web APIs written in Web IDLs,
we developed automatic modeling
of library functions from Web APIs and detect possible misuses of Web APIs
by web applications.  The same technique can support analysis of libraries specified
in TypeScript~\cite{ts}.  Analyzing real-world web applications requires
more scalable analysis than analyzing stand-alone JavaScript programs~\cite{ase15b,ecoop15}.


The baseline analysis is designed to be sound, which means that the properties it
computes should over-approximate the concrete behaviors of the analyzed program.
However, \safe may contain implementation bugs leading to unsound analysis results.
Moreover, some components of \safe may be intentionally unsound, or soundy~\cite{soundy}.
To lessen the burden of analyzing the entire concrete behaviors of programs,
we may use approximate call graphs~\cite{icse13}
from WALA~\cite{wala} to analyze a fraction of them,
or utilize dynamic information statically~\cite{safehybrid}
to prune relatively unrelated code.

\section{Introduction to \safe 2.0}
\label{s:2:2:safe2}
Based on our experiments and experiences with \safe 1.0,
we now release \safe 2.0, which is aimed to be a playground for
advanced research in JavaScript web applications.
Thus, we intentionally designed it to be light-weight, highly parametric, and modular.

The important changes from \safe 1.0 include the following:
\begin{itemize}
\item \safe 2.0 has been tested using Test262, the official ECMAScript (ECMA-262) conformance suite.
\item \safe 2.0 now uses sbt instead of ant to build the framework.
\item \safe 2.0 provides a library of abstract domains that supports
parameterization and high-level specification of abstract semantics on them.
\item Most Java source files are replaced by Scala code and the only Java source code remained is the generated parser code.
\item Several components from \safe 1.0 may not be integrated into \safe 2.0. Such components include interpreter, concolic testing, clone detector, clone refactoring, TypeScript support, Web API misuse detector, and several abstract domains like the string automata domain.
\end{itemize}

We have the following roadmap for \safe 2.0:
\begin{itemize}
\item \safe 2.0 will make monthly updates.
\item The next update will include a \safe document, browser benchmarks, and more Test262 tests.
\item We plan to support some missing features from \safe 1.0 incrementally such as a bug detector, DOM modeling, and jQuery analysis.
\item Future versions of \safe 2.0 will address various analysis techniques, dynamic features of web applications, event handling, modeling framework, compositional analysis, and selective sensitivity among others.
\end{itemize}

\section{Audience}
This document is for users of \safe (Scalable Analysis Framework for ECMAScript)
2.0, a scalable and pluggable analysis framework for JavaScript web applications.
General information on the \safe project is available at an invited talk at ICFP 2016~\cite{safeicfp16}:
\begin{center}
  \url{https://www.youtube.com/watch?v=gEU9utf0sxE}
\end{center}
and the source code and publications are available at:
\begin{center}
  \url{https://github.com/sukyoung/safe}
\end{center}
For more information, please contact the main developers of \safe
at \texttt{safe [at] plrg.kaist.ac.kr}.

SAFE has been used by:
\begin{itemize}
\itemsep-.2em
\item JSAI~\cite{jsai} @ UCSB
\item ROSAEC~\cite{rosaec} @ Seoul National University
\item K framework~\cite{kjs} @ UIUC
\item Ken Cheung~\cite{emse16} @ HKUST
\item Web-based vulnerability detection~\cite{oracle} @ Oracle
\item Tizen~\cite{tizen} @ Linux Foundation
\end{itemize}

\section{Contributors}
The current developers of SAFE 2.0 are as follows:
\begin{itemize}
\itemsep-.2em
\item Jihyeok Park
\item Youngseo Choi
\item Jaemin Hong
\item Joonyoung Park
\item Sukyoung Ryu
\end{itemize}
and the following have contributed to the source code:
\begin{itemize}
\itemsep-.1em
\item Yeonhee Ryou (SAFE 2.0 core)
\item Minsoo Kim (Built-in function modeling)
\item PLRG @ KAIST and our colleagues in S-Core and Samsung Electronics (SAFE 1.0)
\end{itemize}


\section{License}
The \safe source code is released under the BSD license:
\begin{center}
\url{github.com/sukyoung/safe/blob/master/LICENSE}
\end{center}


\section{A sample use of \safe}
\label{sec:usage-example}
Let us consider a very simple JavaScript program stored in a file name ``{sample.js}'' located
in the current directory:
\begin{verbatim}
with({a: 1}) {a = 2;}
\end{verbatim}
Then, one can see how \safe desugars the \texttt{with} statement by the command below:
\begin{verbatim}
safe astRewrite sample.js
\end{verbatim}
which shows an output like the following:
{\small
\begin{verbatim}
The command 'astRewrite' took 178 ms.
  {
    <>alpha<>1 = <>Global<>toObject({
      a : 1
    });
    ("a" in <>alpha<>1 ? <>alpha<>1.a = 2 : a = 2);
  }
\end{verbatim}
}
\noindent
where the names prefixed by \verb!<>! are generated by \safe.
The \safe analysis is performed on control flow graphs of programs,
which can be built by the command below:
\begin{verbatim}
safe cfgBuild sample.js
\end{verbatim}
\noindent
resulting an output as follows:
{\small
\begin{verbatim}
The command 'cfgBuild' took 492 ms.
function[0] top-level {
  Entry[-1] -> [0]
  Block[0] -> [2], [1], ExitExc
    [0] noop(StartOfFile)
    [1] <>new1<>1 := alloc() @ #1
    [2] <>new1<>1["a"] := 1
    [3] <>Global<>ignore1 :=
          <>Global<>toObject(<>new1<>1) @ #2
    [4] <>alpha<>2 := <>Global<>ignore1
  Block[1] -> [3], ExitExc
    [0] assert("a" in <>alpha<>2)
    [1] <>obj<>3 := <>Global<>toObject(<>alpha<>2) @ #3
    [2] <>obj<>3["a"] := 2
    [3] <>Global<>ignore2 := <>obj<>3["a"]
  Block[2] -> [3], ExitExc
    [0] assert(! "a" in <>alpha<>2)
    [1] a := 2
    [2] <>Global<>ignore2 := 2
  Block[3] -> Exit
    [0] noop(EndOfFile)
  Exit[-2]
  ExitExc[-3]
}
\end{verbatim}
}
\noindent
Finally, the following command:
\begin{verbatim}
safe analyze sample.js
\end{verbatim}
analyzes the JavaScript program in the file and shows the analysis results:
{\small
\begin{verbatim}
** heap **
...
#1 -> [[Class]] : "Object"
      [[Extensible]] : true
      [[Prototype]] : #Object.prototype
      "a" -> [ttt] 2
      Set(a)
...

- # of iteration: 6
- # of user functions: 1
- # of touched blocks: 6
    user blocks: 6
    modeling blocks: 0
- # of instructions: 13
\end{verbatim}
}
