\section{Related Work}\label{sec:related}

Several researchers have introduced analysis technique to utilize dynamic
analysis for static analysis in three different ways: combined analysis,
automatic modeling, and pruning states.


\paragraph{Combined Analysis}

The most related previous work is the combined analysis that utilizes dynamic
analysis during Java static analysis introduced by \citet{concerto}.  They
proved that their combined analysis is sound and showed that it could
significantly improve the precision and performance of Java static analysis by
evaluating their tool, \concerto.  However, their approach have several
limitations compared to the dynamic shortcut.  First, they syntactically divides
a given program to \textit{applications} parts for static analysis and
\textit{frameworks} parts for dynamic analysis.  Thus, it is impossible to
freely switching between static and dynamic and even impossible to perform both
of static and dynamic analysis to the same program part in different contexts.
Besides, they introduced \textit{mostly-concrete interpretation} similar with
our sealed symbolic execution.  However, it supports only a special
\textit{unknown} value that represents any possible value.  Thus, it cannot
preserves the precision of complex abstract domains~\cite{revisit-recency,
regex, weaklySPE} frequently used in JavaScript static analysis.  On the other
hand, the sealed symbolic execution automatically detects whether the abstract
semantics are required for abstract values.  Finally, \concerto preserves the
soundness when the program satisfies the \textit{state separation hypothesis}.
They assume that states of application and framework parts do not interrogated
or manipulated by each other.  While it is reasonable for static analysis of
Java applications using external libraries, the assumption is not satisfied for
JavaScript programs in general.  Unlike their approach, our approach does not
have any assumption between static and dynamic analysis parts.


\paragraph{Automatic Modeling}
For static analysis of JavaScript programs, modeling behaviors of built-in
libraries or host-dependent functions is required because they are opaque
codes.  Since manual modeling is error-prone, tedious, and labor-intensive,
several researchers~\cite{safewapi, safets} utilize type information to
automatically model their behaviors.  However, type information is too imprecise
to reflect the detailed semantics and is difficult to represent the
side-effects.  To alleviate the problem, \citet{mimic} introduced the technique
to infer JavaScript codes for opaque codes using concrete execution.  They
captured the effects of opaque codes on user objects by collecting partial
execution traces and synthesized JavaScript codes based on the extracted
behaviors.  They also leveraged ES6 \jscode{Proxy} objects to capture the
effects on user objects.  Instead of synthesizing JavaScript codes,
\citet{opaque-model} presented a \textit{Sample-Run-Abstract (SRA)} approach for
on-demand modeling thus they focused on the current abstract state during static
analysis.  It first \textit{samples} concrete states in a well-distributed way,
\textit{runs} each sampled state on a JavaScript engine, and \textit{abstract}
executions results.  However, all of previous works sacrifice the soundness of
static analysis.  On the other hand, while the dynamic shortcut is not
applicable for all invocation of opaque functions, it is sound if it is
applicable.


\paragraph{Pruning Analysis Scopes}
Another approach to utilize dynamic analysis for JavaScript static analysis is
to prune the scope of analysis.  \citet{determinacy} proposed dynamic
determinacy analysis that infer determinacy facts via concrete executions that a
variable or expression at a given location holds the same value in all possible
execution of the program.  They specialized target source codes with determinacy
facts so that static analyses could benefit from eliminations of dynamic code
generation (e.g. \jscode{eval}) and constant property names.  \citet{blended}
introduced \textit{blended taint analysis} that specializes JavaScript dynamic
language features such as dynamic code generation or variadic function calls.
They first performs dynamic analysis to collect traces with concrete values used
in dynamic language features and restrict the semantics of features based on the
collected traces during static analysis.  \citet{battles, eha} utilizes three
different points to reduce analysis scopes: initial states, dynamically loaded
files, and event handlers.  They dumped the initial states from a specific web
browser to focus on analyzing the behaviors of web browsers running on the
browser.  Then, they collected paths of dynamically loaded files via concrete
executions and utilized the path information in static analysis.  For event
handlers, they intentionally analyzed partial execution flows using concrete
user events.  They collected concrete states for each entry of event handler
during dynamic analysis, merged them to a single abstract state, and analyzed
the event handler with the abstract state.  Unfortunately, all of them does not
preserve soundness of static analysis unlike the dynamic shortcut.


% \subsection{Abstract Counting}
% 
% \begin{itemize}
%   \item Improving flow analyses via $\Gamma$CFA: Abstract garbage collection and
%     counting~\cite{abstract-gc-counting}
%   \item Revisiting recency abstraction for JavaScript: towards an intuitive,
%     compositional, and efficient heap abstraction~\cite{revisit-recency}
% \end{itemize}

% \subsection{Access Analysis}
% 
% \begin{itemize}
%   \item Access Analysis-Based Tight Localization of Abstract
%     Memories~\cite{func-local}
%   \item Design and implementation of sparse global analyses for C-like
%     languages~\cite{sparse}
% \end{itemize}
