\section{Motivation}\label{sec:motivation}

\begin{figure}[t]
  \centering
  \begin{subfigure}[t]{0.5\textwidth}
    \begin{lstlisting}[style=myJSstyle]
function concat() {
  var length = arguments.length;
  if (!length) return [];
  var args = Array(length - 1),
      array = arguments[0],
      index = length;
  while (index--)
    args[index-1] = arguments[index];
  return arrayPush(
    isArray(array) ? copyArray(array) : [array],
    baseFlatten(args, 1));
}
    \end{lstlisting}
    \vspace*{-1em}
    \caption{Lodash's \jscode{concat} function.}
    \label{fig:concat}
  \end{subfigure}
  \begin{subfigure}[t]{0.5\textwidth}
    \begin{lstlisting}[style=myJSstyle,firstnumber=13]
function changeCountry(G) {
  ...
  if (G.selectedVal === "US" && state) {
    // deterministic arguments of `concat`
    state.items = _.concat([["Other", "Other"]],
      WebinarBase.questions.state.items);
    state.selectedVal = _.head(_.head(C.items));
  }
}
    \end{lstlisting}
    \vspace*{-1em}
    \caption{Load the list of states of the United States.}
    \label{fig:changeCountry}
  \end{subfigure}
  \begin{subfigure}[t]{0.5\textwidth}
    \begin{lstlisting}[style=myJSstyle,firstnumber=22]
function getData(e) {
  var option = ... // option of server connection
  post(option).then(function(e) {
    if (e.total_records && e.total_records > 0) {
      // non-deterministic arguments of `concat`
      this.pastEvents =
        _.concat(this.pastEvents, e.events);
      this.total = e.total_records;
    } else this.noPastData = !0
  })
}
    \end{lstlisting}
    \vspace*{-1em}
    \caption{Load more Zoom events from the server.}
    \label{fig:getData}
  \end{subfigure}
  \vspace*{-1em}
  \caption{Motivating Example: Excerpts from Lodash library and JavaScript codes
  in \code{zoom.us} site.}
  \label{fig:example}
\end{figure}

In this section, we will explain the motivation of the combined analysis for
JavaScript programs with a simple motivating example described in
Figure~\ref{fig:example}.  It shows how to utilize dynamic analysis during
static analysis using lazy concrete execution to increase performance and
precision of static analysis.

For a motivating example, we excerpt the \jscode{concat} function in
Figure~\ref{fig:concat} from Lodash library~\cite{lodash} (v4.17.20), which is
the most popular npm package~\footnote{https://www.npmjs.com/browse/depended}
and \inred{124,562} npm packages have dependency with it.  The \jscode{concat}
function creates a new array concatenating given arrays or values.  It first
checks the length of arguments in line 2-3. Then, it stores the first argument
to \jscode{array} in line 5 and copies the remaining arguments to \jscode{args} in
line 7-8.  Finally, it creates a new array by copying the given array via
\jscode{copyArray} or initializing with a single value in line 10, and pushes
each element of \jscode{args} after flattening it via \jscode{baseFlatten} in
line 11.

For applications of Lodash, we excerpt two functions \jscode{changeCountry} in
Figure~\ref{fig:changeCountry} and \jscode{getData} in Figure~\ref{fig:getData}
from the \code{zoom.us}~\cite{zoom} site.  The website \code{zoom.us} is
homepage of Zoom, which is a videotelephony software program developed by Zooom
Video Communications and it is ranked as \inred{16th} popular web site accroding
to Alexa~\footnote{https://www.alexa.com/siteinfo/zoom.us} in November 2020.

When the given arguments of a function invokation are concrete values, we can
perform dynamic analysis instead of static analysis. For example, The
\jscode{changeCountry} function is invoked when a country different with the
current one is selected in registration of Zoom meetings.  When the ``United
States of America'' (USA) is selected, it calls the \jscode{concat} function
with two pre-defined concrete values as arguments to load USA state information
in line 17-18.  The first argument is an array literal \jscode{[["Other",
"Other"]]} and the second one is an array of pairs of abbreviations and names of
USA states defined as follows:
\begin{lstlisting}[style=myJSstyle,numbers=none]
WebinarBase.questions.state.items =
  [["AL","Alabama"], ..., ["WY", "Wyoming"]]
\end{lstlisting}
Moreover, \jscode{this} value is also a concrete value, the Lodash top-level
object \jscode{\_}.  Thus, we could perform dynamic analysis by invoking the
\jscode{concat} funciton with \jscode{\_} as \jscode{this} value and above two
concrete values as arguments.  It increases performance of static analysis by
skipping the analysis of function call in line 17-18 and utilizing the result of
dynamic analysis.

Dynamic analysis is still applicable using lazy concrete execution even if the
arguments are not concrete values.  The \jscode{getData} function is another
part to use \jscode{concat} function but we can only partially apply dynamic
analysis in this case.  It is invoked when loading more Zoom events in
``Webinars \& Events'' page.  At the first time, initial eight events are stored
in \jscode{this.pastEvents}.  For each click of the ``Load More'' button, the
\jscode{getData} sends a POST request to the server and receives additional
event information \jscode{e} in line 24.  Then, eight events in
\jscode{e.events} are appended to \jscode{this.pastEvents} using the
\jscode{concat} function in line 27-28.  In this case, arguments are not
deterministic but dependent on the given data from the server.  However, we know
that the number of arguments are still deterministically defined as \jscode{2}
and they are array objects.  Thus, we just invoke the funciton \jscode{concat}
with \jscode{\_} as \jscode{this} value and two abstract arrays as arguments for
dynamic analysis.  Then, dynamic analysis is successfully performed in line 2-8
because it just checks the length of \jscode{arguments} and passes its elements
to \jscode{array} and \jscode{args}.  Moreover, \jscode{isArray(array)} is also
able to be evaluated using concrete semantics because we know that \code{array}
is an abstract array.  Only copying via \jscode{copyArray}, flattening via
\jscode{baseFlatten}, and pushing via \jscode{arrayPush} utilize the abstract
semantics.  This is the core idea of lazy concrete execution to maximize the
part of dynamic analysis during static analysis.

In the remaining section we formally define the combined analysis with lazy
concrete execution in Section~\ref{sec:formal}.  Then, we explain how to define
the combined analysis for JavaScript programs in Section~\ref{sec:javascript}.
