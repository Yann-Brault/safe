\section{Dynamic Shortcuts}\label{sec:formal}
In this section, we formally define static analysis using dynamic shortcuts by
introducing sealed symbolic execution in the abstract interpretation framework.
We extend the formalization of abstract interpretation of \citet{abs-interp-1977,
abs-interp-1992} and views-based analysis sensitivity of \citet{sens-toplas}.
For dynamic shortcuts, we define sealed symbolic execution with a
sealed symbolic domain and abstract instantiation maps.  To combine
sensitive abstract interpretation and sealed symbolic execution, we define
a combined domain of sensitive abstract domain and sealed symbolic domain and
explain it with a simple example. Finally, We prove the soundness and
termination property of abstract interpretation using the combined domain.


\subsection{Concrete Semantics}

We define a program $\prog$ as a state transition system $(\stset, \trans,
\istset)$.  A program starts with an initial state in $\istset$ and the
transition relation $\trans \subseteq \stset \times \stset$ describes how states
are transformed to other states.  A \textit{collecting semantics} $\sem{\prog} =
\{ \st \in \stset \mid \ist~\in~\istset \wedge \ist \trans^* \st \}$ consists of
reachable states from initial states of the program $\prog$.  We can compute
it using a \textit{transfer function} $\transfer: \dom \rightarrow \dom$ as
follows:
\[
  \sem{\prog} = \underset{n \rightarrow \infty}{\lim}{\transfer^n(\ielem)}\\
  \qquad
  \transfer(\elem) = \elem \join \step(\elem)\\
\]
where the \textit{concrete domain} $\dom = \powerset{\stset}$ is a complete lattice
with $\cup$, $\cap$, and $\subseteq$ as its join($\join$), meet($\meet$), and
partial order($\order$) operators.  The set of states $\ielem$ denotes the
initial states $\istset$.  The \textit{one-step execution} $\step: \dom
\rightarrow \dom$ transforms states using the transition relation $\trans$:
$\step(\elem) = \{ \st' \mid \st \in \elem \wedge \st \trans \st' \}$.

\begin{figure}[t]
  \[
    \begin{array}{r@{~}l@{~}c@{~}l}
      \labdot{0} & \kwif \; (\; \varx \geq 0 \;) & \labdot{1} & \varx = \varx ;\\
                 & \kwelse & \labdot{2} & \varx = -\varx ;\\
      \labdot{3} & \varx = -\varx ; & \labdot{4} \\
    \end{array}
  \]
  \vspace*{-1em}
  \caption{Negation of the absolute value of $\varx$}
  \label{fig:running-example}
\end{figure}

For example, the code in Figure~\ref{fig:running-example} is a simple program
that calculates the negation of the absolute value of the variable $\varx$.
States are pairs of labels and integers stored in $\varx$: $\stset = \labset
\times \mathbb{N}$.  Assume that the initial states are $\istset = \{ (\lab_0,
-42) \}$, which denotes that the program starts at $\lab_0$
with the variable $\varx$ of value $-42$.
Then, it executes with the following trace:
\[
  (\lab_0, -42) \trans (\lab_2, -42) \trans (\lab_3, 42) \trans (\lab_4, -42)
\]


\subsection{Abstract Interpretation}\label{sec:ai}
Abstract interpretation~\cite{abs-interp-1977, abs-interp-1992}
over-approximates the transfer function $\transfer$ as an \textit{abstract transfer
function} $\abstransfer: \absdom \rightarrow \absdom$ to get an
\textit{abstract semantics} $\abssem{\prog}$ in finite iterations as follows:
\[
    \abssem{\prog} = \underset{n \rightarrow
    \infty}{\lim}{(\abstransfer)^n(\iabselem)}\\
\]
We define a \textit{state abstraction} $\dom \galois{\alpha}{\gamma} \absdom$ as
a Galois connection between the concrete domain $\dom$ and an abstract domain
$\absdom$ with a \textit{concretization function} $\gamma$ and an
\textit{abstraction function} $\alpha$.  The initial abstract state $\iabselem
\in \absdom$ represents an abstraction of the initial state set: $\ielem
\subseteq \gamma(\iabselem)$.  The abstract transfer function $\abstransfer:
\absdom \rightarrow \absdom$ is defined as $\abstransfer(\abselem) = \abselem
\join \absstep(\abselem)$ with an \textit{abstract one-step execution}
$\absstep: \absdom \rightarrow \absdom$.  For a sound state abstraction, the
join operator and the abstract one-step execution should satisfy the following
conditions:
\begin{align}
  \forall \abselem_0, \abselem_1 \in \absdom & . \; \gamma(\abselem_0) \cup
  \gamma(\abselem_1) \subseteq \gamma(\abselem_0 \join
  \abselem_1) \label{equ:sound-join}\\
  \forall \abselem \in \absdom & . \; \step \circ \gamma(\abselem) \subseteq
  \gamma \circ \absstep(\abselem) \label{equ:sound-step}
\end{align}

A simple example abstract domain is $\absdom_\pm = \powerset{\{ -, +, 0 \}}$ with
set operators as domain operators; $-$ denotes negative integers, $+$ positive
integers, and $0$ zero.  Assume that we analyze the code in
Figure~\ref{fig:running-example} with the abstract domain and the initial abstract state $\iabselem =
\{ - \}$. Then, the analysis result is $\{ -, + \}$ because $\varx$ can
have a positive value by executing $\varx = -\varx$ but there is no
way for $\varx$ to have $0$ in this program.


\subsection{Analysis Sensitivity}\label{sec:sens}

Abstract interpretation is often defined with \textit{analysis sensitivity} to
increase the precision of static analysis.  A sensitive abstract domain
$\sabsdom: \viewset \rightarrow \absdom$ is defined with a \textit{view
abstraction} $\viewmap: \viewset \rightarrow \dom$ that provides multiple points
of views for reachable states during static analysis.  It maps a finite number
of views $\viewset$ to sets of states $\dom$. Each view $\view \in \viewset$
represents a set of states $\viewmap(\view)$ and each state is included
in a unique view: $\forall \st \in \stset. \; \st \in \viewmap(\view)
\Rightarrow \forall \view' \in \viewset. \st \in \viewmap(\view') \Rightarrow \view = \view'$.
A \textit{sensitive state
abstraction} $\dom \galois{\alpha_\viewmap}{\gamma_\viewmap} \sabsdom$ is a
Galois connection between the concrete domain $\dom$ and the sensitive abstract
domain $\sabsdom$ with the following concretization function:
\[
  \sgamma(\sabselem) = \underset{\view \in \viewset}{\bigcup}
  {\viewmap(\view) \cap \gamma \circ \sabselem(\view)}
\]

With analysis sensitivities, the abstract one-step execution $\sabsstep:
\sabsdom \rightarrow \sabsdom$ is defined as follows:
\[
  \sabsstep(\sabselem) = \lambda \view \in \viewset. \; \underset{\view' \in
  \viewset}{\bigjoin}{\viewtrans{\view'}{\view} \circ \sabselem(\view')}
\]
where $\viewtrans{\view'}{\view}: \absdom \rightarrow \absdom$ is an abstract
semantics of a \textit{view transition} from a view $\view'$ to another view
$\view$.  It should satisfy the following condition for the soundness of the
analysis:
\[
  \forall \abselem \in \absdom. \; \step(\gamma(\abselem) \cap \viewmap(\view'))
  \cap \viewmap(\view) \subseteq \gamma \circ
  \viewtrans{\view'}{\view}(\abselem)
\]

One of the most widely-used analysis sensitivity is \textit{flow sensitivity}
defined with a flow-sensitive view abstraction $\fsviewmap: \labset
\rightarrow \dom$ where:
\[
  \forall \lab\in\labset. \; \fsviewmap(\lab) = \{ \st \mid \st = (\lab, \_) \}
\]
If we apply the flow sensitivity for the above example with the initial abstract
state $[ \lab_0 \mapsto \{ -, 0, + \} ]$, the analysis result is as follows:
\[
  \begin{array}{|c||c|c|c|c|c|}\hline
    \labset & \lab_0 & \lab_1 & \lab_2 & \lab_3 & \lab_4\\\hline
    \absdom_\pm & -, 0, + & 0, + & - & 0, + & -, 0\\\hline
  \end{array}
\]


\subsection{Sealed Symbolic Execution}

We define \textit{sealed symbolic execution} by extending the transition
relation $\trans$ as a symbolic transition relation $\symbtrans$ on symbolic
states.  First, we extend concrete states $\stset$ to symbolic states
$\symbstset$ by extending values $\valset$ with \textit{sealed symbolic values}
$\symbset$.  We also define the symbolic transition relation $\symbtrans
\subseteq \symbstset \times \symbstset$. We use the notation $\symbtrans^k$
for $k$ repetition of $\symbtrans$, and write $\symbst \symbtrans \excst$ when
$\symbst$ does not have any symbolic transitions to other sealed symbolic
states.  We define the validity of sealed symbolic execution as follows:
\begin{definition}[Validity]\label{def:valid-symbtrans}
  The symbolic transition relation is \textit{valid} when the following
  condition is satisfied for any sealed symbolic states $\symbst$ and
  $\symbst'$:
  \[
    \symbst \symbtrans \symbst' \Leftrightarrow
    \forall \imap \in \imapset. \;
    \{ \st' \mid \instant{\symbst}{\imap} \trans \st' \}
    = \{ \instant{\symbst'}{\imap} \}
  \]
  where $\imapset: \symbset \rightarrow \valset$ represent \textit{instantiation
  maps} from symbolic values to concrete values, and $\instant{\symbst}{\imap}$
  denotes a state produced by replacing each symbolic value $\symb$ in
  $\symbst$ with its
  corresponding value $\imap(\symb)$ using the instantiation map $\imap \in
  \imapset$.
\end{definition}

Sealed symbolic execution is different from traditional
symbolic execution~\cite{symbolic} in that it supports only sealed symbolic
values instead of symbolic expressions and path constraints.  For example, the
following trace represents traditional symbolic execution of the running
example in Figure~\ref{fig:running-example}:
{
\small
\[
  \begin{array}{r@{~}c@{~}c@{~}c@{~}r@{~}c@{~}r}
    &&(\lab_1, \symb)[\symb \!\geq\! 0]
    &\trans& (\lab_3, \phantom{-}\symb)[\symb \!\geq\! 0]
    &\trans& (\lab_4, -\symb)[\symb \!\geq\! 0]
    \vspace*{-0.5em}\\
    &\rutrans&
    \vspace*{-0.5em}\\
    (\lab_0, \symb)[\varnothing]
    \vspace*{-0.5em}\\
    &\rdtrans&
    \vspace*{-0.5em}\\
    &&(\lab_2, \symb)[\symb \!<\! 0]
    &\trans& (\lab_3, -\symb)[\symb \!<\! 0]
    &\trans& (\lab_4, \phantom{-}\symb)[\symb \!<\! 0]\\
  \end{array}
\]
}
It first assigns a symbolic value $\symb$ to the variable $\varx$ at $\lab_0$.
For the conditional branch, it creates two symbolic states with
different path conditions $\symb \geq 0$ and $\symb < 0$ for true and false
branches, respectively.  After executing statements $\varx = \varx$ and $\varx =
-\varx$, the variable $\varx$ stores symbolic expressions $\symb$ and $-\symb$
at $\lab_3$, respectively. Similarly, $\varx$  stores $-\symb$ and $\symb$ at $\lab_4$.
However, sealed symbolic execution stops at $\lab_0$ as follows:
\[
  (\lab_0, \symb) \; \symbtrans \; \excst
\]
because the branch requires the actual value of the symbolic value $\symb$.

To define an abstract domain that contains sealed symbolic states, we define
\textit{abstract instantiation maps} $\absimapset: \symbset \rightarrow
\absvalset$ from symbolic values to abstract values.  Its concretization
function $\imapgamma: \absimapset \rightarrow \powerset{\imapset}$ is defined
with the concretization function $\valgamma: \absvalset \rightarrow
\powerset{\valset}$ for values as follows:
\[
  \imapgamma(\absimap) = \{
    \imap \mid \forall \symb \in \symbset. \;
    \imap(\symb) \in \gamma \circ \absimap(\symb)
  \}
\]
The instantiation of a given sealed symbolic state $\symbst \in \symbstset$ with
an abstract instantiation map $\absimap \in \absimapset$ is defined as follows:
\[
  \instant{\symbst}{\absimap} = \{ \instant{\symbst}{\imap} \mid \imap \in
  \imapgamma(\absimap) \}
\]
Now, we define a \textit{sealed symbolic domain} as follows:

\begin{definition}[Sealed Symbolic Domain]\label{def:symbdom}
  A \textit{sealed symbolic domain} $\symbdom: \powerset{\absimapset \times
  \symbstset}$ is defined with the concretization function
  $\symbgamma: \symbdom \rightarrow \dom$ and the sealed symbolic one-step execution
  $\symbstep: \symbdom \rightarrow \symbdom$ such that
  \begin{align}
    \symbgamma(\symbelem) &=
    \bigcup \{ \instant{\symbst}{\absimap} \mid (\absimap, \symbst) \in
    \symbelem\}\\
    \symbstep(\symbelem) &= \{ (\absimap, \symbst') \mid (\absimap, \symbst)
    \in \symbelem \wedge \symbst \symbtrans \symbst' \}
  \end{align}
\end{definition}
