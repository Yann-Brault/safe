\begin{abstract}
JavaScript has become one of the most widely used programming languages for web
development, server-side programming, and even micro-controllers for IoT.
However, its extremely functional and dynamic features degrade the scalability and precision
of static analysis.  Moreover, the variety of built-in functions and host
environments requires excessive manual modeling of their behaviors.  To
alleviate these problems, researchers have proposed various ways to
leverage dynamic analysis during JavaScript static analysis.  However,
they do not fully utilize the high performance of dynamic analysis and
often sacrifice the soundness of static analysis.

In this paper, we propose a novel technique to take advantage of the
high performance of dynamic analysis for JavaScript static analysis
in a sound way by using \textit{dynamic shortcuts}.  A dynamic shortcut
consists of three parts: 1) converting an abstract state to its corresponding
\textit{sealed symbolic state}, 2) performing \textit{sealed symbolic execution}
on the sealed symbolic state, and 3) converting a sealed symbolic state
to its corresponding abstract state. By taking dynamic shortcuts
during static analysis, we can significantly improve the analysis
scalability and precision by using highly-optimized commercial JavaScript engines
and lessen the modeling efforts by performing sealed symbolic execution for opaque code.
We formally define static analysis using dynamic shortcuts in the abstract
interpretation framework.  We actualize the technique via $\tool$, an
extended combination of SAFE and Jalangi, a static analyzer and a
dynamic analyzer, respectively.  We evaluated $\tool$ using
269 official tests of Lodash 4 library.
Our experiment shows that $\tool$ is 6.30$\x$ faster than the
baseline static analyzer, and it improves the precision to
detect 6 more dead branches on average by using sealed symbolic execution for
12 opaque functions.
\end{abstract}
