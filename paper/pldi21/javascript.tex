\section{Dynamic Shortcuts for JavaScript}\label{sec:javascript}
In this section, we introduce the core language of JavaScript that supports
first-class functions, open objects, and first-class property names, and define
sealed symbolic execution of the core language for dynamic shortcuts.
Due to the space limitation, we present the main design of the
language in this paper and refer the interested readers to a companion report~\cite{report}.

\subsection{Core Language of JavaScript}

\[
  \begin{array}{ll@{~}c@{~}l}
    \text{Programs} & \prog &::=& (\lab: \inst)^*\\

    \text{Labels} & \lab &\in& \labset\\

    \text{Instructions} & \inst &::=&
    \refer = \expr \mid
    \refer = \kwobj \mid
    \refer = \expr ( \expr ) \mid
    \kwret \; \expr \mid
    \kwif \; \expr \; \lab\\

    \text{References} & \refer &::=&
    x \mid
    \expr [ \expr ]\\

    \text{Expressions} & \expr &::=&
    \pval \mid
    \lambda x. \; \lab \mid
    \refer \mid
    \op(\expr^*)\\
  \end{array}
\]

A program $\prog$ is a sequence of labeled instructions. An instruction $\inst$
is an expression assignment, an object creation, a function call, a return
instruction, or a branch.  A reference $\refer$ is a variable or a property
access of an object.  An expression $\expr$ is a primitive, a lambda function, a
reference, or an operation between other expressions.

\[
  \begin{array}{lr@{~}c@{~}l@{~}c@{~}l}
    \text{States} & \st &\in& \stset &=& \labset \times \memset \times
    \ctxtset \times \eaddrset\\
    \text{Memories} & \mem &\in& \memset &=& \locset \finmap \valset\\
    \text{Contexts} & \ctxt &\in& \ctxtset &=& \eaddrset \finmap (\eaddrset
    \times \labset \times \locset)\\
    \text{Locations} & \loc &\in& \locset &=& (\eaddrset \times \varset) \uplus
    (\oaddrset \times \strset)\\
    \text{Values} & \val &\in& \valset &=& \pvalset \uplus \oaddrset \uplus
    \fvalset\\
    \text{Primitives} & \pval &\in& \pvalset &=& \strset \uplus \cdots\\
    \text{Addresses} & \addr &\in& \addrset &=& \eaddrset \uplus \oaddrset\\
    \text{Functions} & \fval{x}{\lab} &\in& \fvalset &=& \varset \times
    \labset\\
  \end{array}
\]

States $\stset$ consist of labels $\labset$, memories $\memset$, contexts
$\ctxtset$, and environment addresses $\eaddrset$.  A memory $\mem \in \memset$
is a finite mapping from locations to values.  A context $\ctxt \in \ctxtset$ is
a finite mapping from environment addresses to tuple of environment addresses,
return labels, and left-hand side locations.  A location $\loc \in \locset$ is a
variable or an object property; a variable location consists of an environment
address and its name, and an object property location consists of an object
address and a string value.  A value $\val \in \valset$ is a primitive, an
address, or a function value.  An address $\addr \in \addrset$ is an environment
address or an object address.  A function value $\fval{x}{\lab} \in \fvalset$
consists of a parameter name and a body label.  In the core language, the closed
scoping is used for functions for brevity, thus only parameters and local
variables are accessible in a function body.

We formulate the concrete semantics of the core language.  The transition
relation between concrete states is defined with the semantics of references and
expressions using two different forms \fbox{$\referrule{\st}{\refer}{\loc}$} and
\fbox{$\exprrule{\st}{\expr}{\val}$}, respectively.  The initial states are
$\istset = \{ (\ilab, \varnothing, \epsilon, \tladdr) \}$ where $\ilab$ denotes
the initial label, $\epsilon$ empty map, and $\tladdr$ the top-level environment address.

\subsection{Abstract Semantics}
In the abstract semantics of the core language, we use the flow sensitivity with a
flow sensitive view abstraction $\fsviewmap: \labset \rightarrow \dom$ that
discriminates states using their labels: $\forall \lab \in \labset. \;
\fsviewmap(\lab) = \{ \st \in \stset \mid \st = (\lab, \_, \_, \_) \}$. Thus, the
sensitive abstract domain is defined as $\sabsdom = \labset \rightarrow
\absdom$.  We define an abstract state $\abselem \in \absdom$ as a tuple of an
abstract memory, an abstract context, an abstract address, and an
abstract counter as follows:

\[
  \begin{array}{l@{~}r@{~}c@{~}l@{~}c@{~}l}
\text{Abstract states} & \abselem &\in& \absdom &=& \absmemset \times \absctxtset
\times \absaddrset \times \abscountset\\
\text{Abstract memories} & \absmem &\in& \absmemset &=& \abslocset \finmap
\absvalset\\
\text{Abstract locations} & \absloc &\in& \abslocset &=& (\absaddrset \times
\varset) \uplus (\absaddrset \times \strset)\\
\text{Abstract addresses} & \absaddr &\in& \absaddrset &=& \labset\\
\text{Abstract contexts} & \absctxt &\in& \absctxtset &=& \absaddrset \finmap
\powerset{\absaddrset \times \viewset \times \powerset{\abslocset}}\\
\text{Abstract counters} & \abscount &\in& \abscountset &=& \absaddrset
\rightarrow \{ \abszero, \absone, \absmany \}\\
\text{Abstract values} & \absval &\in& \absvalset &=& \powerset{\pvalset
\uplus \absaddrset \uplus \fvalset}\\
  \end{array}
\]

An abstract memory $\absmem \in \absmemset$ is a finite mapping from abstract
locations $\abslocset$ to abstract values $\absvalset$.  Abstract locations
$\abslocset$ are pairs of abstract addresses with variable names or string
values. Abstract addresses $\absaddrset$ are defined with the
\textit{allocation-site abstraction} that partitions concrete addresses
$\addrset$ based on their allocation sites $\labset$.  Abstract contexts
$\absctxtset$ are finite maps from abstract addresses to powersets of triples of
abstract addresses, views, and powerset of abstract locations.  For abstract
counting~\cite{abstract-gc-counting, revisit-recency} in static analysis, we
define abstract counters $\abscountset$ that are mappings from abstract addresses to
their abstract counts representing how many times each abstract address has been
allocated; $\abszero$ denotes that it has never been allocated, $\absone$ once,
and $\absmany$ more than or equal to twice.

We define the semantics of the view transition for the core language.  For abstract
memories, we use the notation $\absmem[L \mapstos \absval]$ to represent the
update of multiple abstract locations in $L$ with the abstract value $\absval$.
It performs the strong update if the abstract address for an abstract location
$(\absaddr, \_) \in L$ is singleton: $\abscount(\absaddr) = \absone$.
Otherwise, it performs the weak update for the analysis soundness.  We use
the increment function $\inc: \abscountset \times \absaddrset \rightarrow
\abscountset$ of the abstract counter defined as follows:
\[
  \inc(\abscount)(\absaddr_0) = \lambda \absaddr \in \absaddrset. \; \left\{
    \begin{array}{ll}
      \absone & \text{if} \; \absaddr = \absaddr_0 \wedge
      \abscount(\absaddr_0) = \abszero\\
      \absmany & \text{if} \; \absaddr = \absaddr_0 \wedge
      \abscount(\absaddr_0) = \absone\\
      \abscount(\absaddr) & \text{otherwise}
    \end{array}
  \right.
\]


\subsection{Sealed Symbolic Execution}

We define the sealed symbolic states by not only extending the concrete values
$\valset$ with symbolic values $\symbset$ but also adding the abstract counters
$\abscountset$ as follows:
\[
  \begin{array}{r@{~}c@{~}l}
    \symbstset &=& \labset \times \memset \times \ctxtset \times \eaddrset
    \times \abscountset\\
    \ctxtset &=& \eaddrset \finmap ((\eaddrset \times \labset \times \locset)
    \uplus \symbset)\\
    \valset &=& \pvalset \uplus \oaddrset \uplus \fvalset \uplus \symbset\\
    \abscountset &=& \oaddrset \rightarrow \{ \abszero, \absone, \absmany \}\\
  \end{array}
\]
Because JavaScript provides open objects, the properties of objects can be dynamically added or deleted.
Moreover, since object properties are string values that can be constructed at run time,
it is difficult for static analysis to know whether strong udpates are possible or not.
To check the possibility of strong updates during sealed symbolic execution,
we augment its states with the abstract counters $\abscountset$.

For each abstract value in a given abstract state,
if the abstract value denotes a single concrete value,
the converter $\asconverter: (\viewset \times
\absdom) \rightarrow (\absimapset \times \symbstset)$
keeps it; otherwise, $\asconverter$ replaces the abstract
value with its unique identifier and maintains the mapping from the
unique identifier to the abstract value to construct an abstract instantiation map.
The opposite converter $\saconverter: (\absimapset \times
  \symbstset)  \rightarrow (\viewset \times \absdom)$
recovers abstract values from their unique identifiers using the abstract instantiation map.
We define the sealed symbolic transition relation $\symbtrans$
only if the next step does not require actual values of any symbolic values.
Otherwise, a given sealed symbolic state does not have any sealed symbolic
transitions to apply.  For example, we add the following rule for the $\kwret$ statement:
\begin{mathpar}
  \inferrule
  {
    \prog(\lab) = \kwret \; \expr\\
    \exprrule{\symbst}{\expr}{\val}\\
    \ctxt(\addr) \in \symbset
  }
  {
    \symbst = (\lab, \mem, \ctxt, \addr, \abscount)\
    \symbtrans \excst
  }
\end{mathpar}
We extend each rule of the concrete semantics to support such behaviors of symbolic
values.
