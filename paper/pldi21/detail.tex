\section{Application: JavaScript}

In this section, we introduce the core language of JavaScript that supports
first-class functions, open objects, and first-class property names, and define
combined analysis of the core language.

\subsection{Core Language of JavaScript}

\begin{figure*}[t]
  \centering

  \fbox{$\st \trans \st$}
  \begin{mathpar}
    \inferrule*[width=0.48\textwidth]
    {
      \prog(\lab) = \refer = \expr\\
      \referrule{\st}{\refer}{\loc}\\
      \exprrule{\st}{\expr}{\val}\\
    }
    {
      \st = (\lab, \mem, \ctxtstack, \addr)
      \trans
      (\labnext(\lab), \mem[\loc \mapsto \val], \ctxtstack, \addr)
    }

    \inferrule*[width=0.48\textwidth]
    {
      \prog(\lab) = \refer = \kwobj\\
      \referrule{\st}{\refer}{\loc}\\
      \addr' = \text{(a fresh object address)}
    }
    {
      \st = (\lab, \mem, \ctxtstack, \addr)
      \trans
      (\labnext(\lab), \mem[\loc \mapsto \addr'], \ctxtstack, \addr)
    }

    \inferrule*[width=0.48\textwidth]
    {
      \prog(\lab) = \refer = \expr_f ( \expr_a )\\
      \referrule{\st}{\refer}{\loc}\\
      \exprrule{\st}{\expr_f}{\fval{x}{\lab_b}}\\
      \exprrule{\st}{\expr_a}{\val_a}\\
      \addr' = \text{(a fresh environment address)}
    }
    {
      \st = (\lab, \mem, \ctxtstack, \addr)
      \trans
      (\lab_b, \mem[(\addr', x) \mapsto \val_a], (\addr, \labnext(\lab), \loc)
      :: \ctxtstack, \addr')
    }

    \inferrule*[width=0.48\textwidth]
    {
      \prog(\lab) = \kwret \; \expr\\
      \exprrule{\st}{\expr}{\val}\\
    }
    {
      \st = (\lab, \mem, (\addr', \lab', \loc) :: \ctxtstack, \addr)
      \trans
      (\lab', \mem[\loc \mapsto \val], \ctxtstack, \addr')
    }

    \inferrule*[width=0.48\textwidth]
    {
      \prog(\lab) = \kwif \; \expr \; \lab'\\
      \exprrule{\st}{\expr}{\kwtrue}\\
    }
    {
      \st = (\lab, \mem, \ctxtstack, \addr)
      \trans
      (\lab', \mem, \ctxtstack, \addr)
    }

    \inferrule*[width=0.48\textwidth]
    {
      \prog(\lab) = \kwif \; \expr \; \lab'\\
      \exprrule{\st}{\expr}{\kwfalse}\\
    }
    {
      \st = (\lab, \mem, \ctxtstack, \addr)
      \trans
      (\labnext(\lab), \mem, \ctxtstack, \addr)
    }
  \end{mathpar}

  \fbox{$\referrule{\st}{\refer}{\loc}$}
  \begin{mathpar}
    \inferrule*[width=0.48\textwidth]
    {}
    {
      \referrule{\st = (\lab, \mem, \ctxtstack, \addr)}{x}{(\addr, x)}\\
    }

    \inferrule*[width=0.48\textwidth]
    {
      \exprrule{\st}{\expr_0}{\addr_0}\\
      \exprrule{\st}{\expr_1}{\val_1}\\
      \val_1 \in \strset\\
    }
    {
      \referrule{\st = (\lab, \mem, \ctxtstack, \addr)}{\expr_0 [ \expr_1
      ]}{(\addr_0, \val_1)}
    }
  \end{mathpar}

  \fbox{$\exprrule{\st}{\expr}{\val}$}
  \begin{mathpar}
    \inferrule*[width=0.48\textwidth]
    {
    }
    {
      \exprrule{\st = (\lab, \mem, \ctxtstack, \addr)}{\pval}{\pval}
    }

    \inferrule*[width=0.48\textwidth]
    {
    }
    {
      \exprrule{\st = (\lab, \mem, \ctxtstack,
      \addr)}{\fval{x}{\lab'}}{\fval{x}{\lab'}}
    }

    \inferrule*[width=0.48\textwidth]
    {
      \referrule{\st}{\refer}{\loc}\\
      \loc \in \mem\\
    }
    {
      \exprrule{\st = (\lab, \mem, \ctxtstack, \addr)}{\refer}{\mem(\loc)}
    }

    \inferrule*[width=0.48\textwidth]
    {
      \referrule{\st}{\refer}{\loc}\\
      \loc \not\in \mem\\
    }
    {
      \exprrule{\st = (\lab, \mem, \ctxtstack, \addr)}{\refer}{\kwundef}
    }

    \inferrule*[width=0.48\textwidth]
    {
      \exprrule{\st}{\expr_1}{\val_1}\\
      \cdots\\
      \exprrule{\st}{\expr_n}{\val_n}\\
    }
    {
      \exprrule{\st = (\lab, \mem, \ctxtstack, \addr)}
      {\op(\expr_1, \cdots, \expr_n)}{\op(\val_1, \cdots, \val_n)}
    }
  \end{mathpar}

  \caption{The transition relation for the core language of JavaScript}
  \label{fig:core-trans-rel}
\end{figure*}

\[
  \begin{array}{ll@{~}c@{~}l}
    \text{Programs} & \prog &::=& (\lab: \inst)^*\\

    \text{Labels} & \lab &\in& \labset\\

    \text{Instructions} & \inst &::=&
    \refer = \expr \mid
    \refer = \kwobj \mid
    \refer = \expr ( \expr ) \mid
    \kwret \; \expr \mid
    \kwif \; \expr \; \lab\\

    \text{References} & \refer &::=&
    x \mid
    \expr [ \expr ]\\

    \text{Expressions} & \expr &::=&
    \pval \mid
    \lambda x. \; \lab \mid
    \refer \mid
    \op(\expr^*)\\
  \end{array}
\]

A program $\prog$ is a sequence of labelled instructions. An instruction $\inst$
is an expression assignment, an object creation, a function call, a return
instruction, or a branch.  An expression $\expr$ is a primitive, a lambda
function, a reference, or an operation between other expressions.  A reference
$\refer$ is a variable or a property access of an object.

\[
  \begin{array}{lr@{~}c@{~}l@{~}c@{~}l}
    \text{States} & \st &\in& \stset &=& \labset \times \memset \times
    \ctxtset^* \times \eaddrset\\
    \text{Memories} & \mem &\in& \memset &=& \locset \rightarrow \valset\\
    \text{Contexts} & \ctxt &\in& \ctxtset &=& \eaddrset \times \labset \times
    \locset\\
    \text{Locations} & \loc &\in& \locset &=& (\eaddrset \times \varset) \uplus
    (\oaddrset \times \strset)\\
    \text{Values} & \val &\in& \valset &=& \pvalset \uplus \oaddrset \uplus
    \fvalset\\
    \text{Primitives} & \pval &\in& \pvalset &=& \strset \uplus \cdots\\
    \text{Addresses} & \addr &\in& \addrset &=& \eaddrset \uplus \oaddrset\\
    \text{Functions} & \fval{x}{\lab} &\in& \fvalset &=& \varset \times
    \labset\\
  \end{array}
\]

States $\stset$ consist of labels $\labset$, memories $\memset$, context stacks
$\ctxtset^*$, and environment addresses.  A memory $\mem \in \memset$ is a
mapping from locations to values.  Context stacks is sequences of contexts
$\ctxtset$ and a context $\ctxt \in \ctxtset$ is a tuple of an environment
address, a return label, and a left-hand side location.  A location $\loc \in
\locset$ is a variable or an object property; a variable location consists of an
environment address and its name, and an object property location consists of an
object address and a string value.  A value $\val \in \valset$ is a primitive,
an address, or a function value.  An address $\addr \in \addrset$ is an
environment address or an object address.  A function value
$\fval{x}{\lab}{\addr} \in \fvalset$ consists a parameter and a body label.  In
the core language, the closed scoping is used for functions for brevity thus
only parameters and local variables are accessible in the function body.

We formulate the concrete semantics of the core language as described in
Figure~\ref{fig:core-trans-rel}.  The transition relation between concrete
states is defined with the semantics of references and expressions using two
different forms \fbox{$\referrule{\st}{\refer}{\loc}$} and
\fbox{$\exprrule{\st}{\expr}{\val}$}, respectively.  The special value
$\kwundef$ denotes an undefined value and it is produced when the program access
an unknown location.


\subsection{Abstract Semantics}

\todo
