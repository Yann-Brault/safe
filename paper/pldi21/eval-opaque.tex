\subsection{Opaque Function Coverage}
To evaluate how much manual modeling efforts of opaque functions
are reduced by dynamic shortcuts, we measured the number of tests where
opaque functions that are analyzed only by dynamic analysis not by
static analysis.  Table~\ref{table:func-replace} summarizes the result.
For \inred{xx} original tests and \inred{xx} abstracted tests, we measured the
number of tests that use only dynamic shortcuts instead of manual modeling
for each JavaScript built-in library function.  For each row,
\textbf{Object} column denotes a built-in object,
\textbf{Function} a function name, and
\inred{\textbf{\# Replaced} the number of tests successfully replacing manual
modeling via dynamic shortcuts with the total number of tests using the target function.
For example, the fourth row in the left side describes that \jscode{Array.prototype.toString} is used in
\inred{139} original tests and \inred{xxx} abstracted tests.
 Among them, \inred{44} original tests \inred{xx} abstracted tests are successfully analyzed
by dynamic shortcuts instead of using the modeling of
\jscode{Array.prototype.toString}.
Moreover, several built-in functions are analyzed by only dynamic shortcuts in
all tests.  Among \inred{49} JavaScript built-in functions, \inred{xx} and
\inred{xx} functions are analyzed by only dynamic shortcut instead of manual
modeling for all original and all abstracted tests, respectively.  Each filled cell
describes fully replaceable cases in Table~\ref{table:func-replace}.
Therefore, dynamic shortcuts effectively lessen
the burden of manual modeling for JavaScript built-in functions.
}
