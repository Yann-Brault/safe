\subsection{Opaque Function Coverage}
To evaluate how much manual modeling efforts of opaque functions
are reduced by dynamic shortcuts, we measured the number of tests where
opaque functions that are analyzed only by dynamic analysis not by
static analysis.  Table~\ref{table:func-replace} summarizes the result.
For 198 original tests and 143 abstracted tests, we measured the
number of tests that use only dynamic shortcuts instead of manual modeling
for each JavaScript built-in library function.  For each row,
\textbf{Object} column denotes a built-in object, \textbf{Function} a function
name, and \textbf{\# Replaced} the number of tests successfully replacing manual
modeling via dynamic shortcuts with the total number of tests using the target function.
For example, the first row in the left side describes that \jscode{Array} is used in
203 original tests and 116 abstracted tests.  Among them, 203 original
tests 90 abstracted tests are successfully analyzed by dynamic shortcuts
instead of using the modeling of \jscode{Array}.  Each filled cell describes
fully replaceable cases in Table~\ref{table:func-replace}.  Therefore, dynamic
shortcuts effectively lessen the burden of manual modeling for JavaScript
built-in functions.  Again, since all the original Lodash 4 tests except one
test is finisehd with a single dynamic shortcut, all the built-in fuctions
except \jscode{Math.floor} and \jscode{Function.prototype.call} are replaceable
for original tests.  For abstracted tests, 12 functions out of 60 built-in
functions are analyzed by only dynamic shortcut instead of manual modeling.
