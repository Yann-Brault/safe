\section{Motivation}\label{sec:motivation}

\begin{figure}[t]
  \centering
  \begin{subfigure}[t]{0.5\textwidth}
    \begin{lstlisting}[style=myJSstyle]
function concat() {
  var length = arguments.length;
  if (!length) return [];
  var array = arguments[0],
      args  = Array(length - 1),
      index = length;
  while (index--)
    args[index-1] = arguments[index];
  return arrayPush(isArray(array) ?
    copyArray(array) : [array],
    baseFlatten(args, 1));
}
    \end{lstlisting}
    \vspace*{-1em}
    \caption{Lodash's \jscode{concat} function}
    \label{fig:concat}
  \end{subfigure}
  \begin{subfigure}[t]{0.5\textwidth}
    \begin{lstlisting}[style=myJSstyle,firstnumber=13]
function changeCountry(G) { ...
  if (G.selectedVal === "US" && state) {
    // deterministic arguments of `concat`
    state.items = _.concat([["Other", "Other"]],
      WebinarBase.questions.state.items);
    state.selectedVal = _.head(_.head(C.items));
  }
}
    \end{lstlisting}
    \vspace*{-1.5em}
    \caption{Call of \jscode{concat} with concrete values}
    \label{fig:changeCountry}
  \end{subfigure}
  \begin{subfigure}[t]{0.5\textwidth}
    \begin{lstlisting}[style=myJSstyle,firstnumber=22]
function getData(e) {
  var option = ... // option for server connection
  post(option).then(function(e) {
    if (e.total_records && e.total_records > 0) {
      // non-deterministic arguments of `concat`
      this.pastEvents =
        _.concat(this.pastEvents, e.events);
      this.total = e.total_records;
    } else this.noPastData = !0
  })
}
    \end{lstlisting}
    \vspace*{-1.5em}
    \caption{Call of \jscode{concat} with abstract values}
    \label{fig:getData}
  \end{subfigure}
  \vspace*{-1em}
  \caption{Lodash library function and its uses in \code{zoom.us}}
  \label{fig:example}
  \vspace*{-1.5em}
\end{figure}

This section explains the motivation of dynamic shortcuts
using real-world examples in Figure~\ref{fig:example}.
We describe their behaviors and explain how we can utilize
dynamic shortcuts during static analysis.

Figure~\ref{fig:example}(a) shows the \jscode{concat} function defined
in Lodash library~\cite{lodash} (v4.17.20); it is
the most popular npm package\footnote{https://www.npmjs.com/browse/depended}
and 131,517 npm packages have a dependency on it.
The \jscode{concat} function creates a new array concatenating given arrays or values.
It first checks the length of \jscode{arguments} on lines 1--3.
Then, it stores the first argument to \jscode{array} on line 4 and
copies the remaining arguments to \jscode{args} on lines 5--8.
On line 9, it checks whether \jscode{array} is an array object using
the built-in function \jscode{isArray}.  If so, it creates a new array
by copying the given array via \jscode{copyArray}; otherwise,
it creates a singleton array \jscode{[array]}.  Finally, it flattens
\jscode{args} via \jscode{baseFlatten} and pushes the result to the
new array on line 11.

Figure~\ref{fig:example}(b) and Figure~\ref{fig:example}(c) show use
cases of the \jscode{concat} function in the \code{zoom.us}~\cite{zoom} site.
It is the homepage of Zoom, a videotelephony software by Zoom
Video Communications and it is ranked as the 15th popular web site according
to Alexa\footnote{https://www.alexa.com/siteinfo/zoom.us} in February 2021.

\paragraph{Dynamic shortcuts with concrete values.}
When a function is called with concrete values, we can perform
dynamic analysis instead of static analysis.
For example, \jscode{changeCountry} in Figure~\ref{fig:example}(b)
is invoked when a user selects a country from a drop-down list in the registration page.
It calls the \jscode{concat} function to update the drop-down list of
states or provinces on lines 16--17.  However, when the user selects ``United States of America,''
which is \jscode{"US"}, two arguments are pre-defined with
deterministic values; the first one is an array literal
\jscode{[["Other", "Other"]]} and the second one is an array of pairs
of abbreviations and names of the states defined as follows: \\[-.5em]
\begin{lstlisting}[style=myJSstyle,numbers=none]
WebinarBase.questions.state.items =
  [["AL","Alabama"], ..., ["WY", "Wyoming"]]
\end{lstlisting}
\noindent\smallskip
Moreover, \jscode{this} also has a concrete value, the Lodash top-level object~\jscode{\_}.
Thus, we can perform dynamic analysis by invoking \jscode{concat} with~\jscode{\_}
as its \jscode{this} value and the above concrete values as arguments. 
By skipping the analysis of the function call on lines 17--18 and
utilizing the result of dynamic analysis, it improved the analysis performance.

\begin{figure}[t]
  \begin{subfigure}{0.23\textwidth}
    \[
      \begin{array}{|c|c|}\hline
        \text{Property} & \text{Value}\\\hline
        \top & \symb_\jscode{evt}\\\hline
        \jscode{"length"} & \symb_\jscode{int}\\\hline
      \end{array}
    \]
    \vspace*{-1em}
    \caption{\jscode{this.pastEvents}}
    \label{fig:pastEvents}
  \end{subfigure}
  \begin{subfigure}{0.23\textwidth}
    \[
      \begin{array}{|c|c|}\hline
        \text{Property} & \text{Value}\\\hline
        \jscode{0} & \symb_\jscode{evt}\\\hline
        \cdots & \cdots\\\hline
        \jscode{7} & \symb_\jscode{evt}\\\hline
        \jscode{"length"} & \jscode{8}\\\hline
      \end{array}
    \]
    \vspace*{-1em}
    \caption{\jscode{e.events}}
    \label{fig:events}
  \end{subfigure}
  \vspace*{-1em}
  \caption{Concrete objects with {\sealed} values}
  \label{fig:sealed}
  \vspace*{-1em}
\end{figure}

\paragraph{Dynamic shortcuts with abstract values.}
Even when a function is called with abstract values, we can still
perform dynamic analysis using {\sealed} execution.
For example, \jscode{getData} in Figure~\ref{fig:example}(c) is invoked
when a user clicks the ``Load More'' button to load more Zoom events in
the ``Webinars \& Events'' page.  It sends a POST
request to a server and receives additional events \jscode{e} on
line 24.  Then, eight events in \jscode{e.events} are appended to
\jscode{this.pastEvents} using \jscode{concat} on lines 27--28.
However, the arguments of \jscode{concat} are not deterministic because 1) the
event list stored in \jscode{this.pastEvents} is continuously grown for each
load and 2) the events stored in \jscode{e.events} are dependent on the
data given from the server.

To perform dynamic analysis with abstract values, we seal abstract values
with {\sealed} values as in Figure~\ref{fig:sealed}.  Two {\sealed}
values $\symbevt$ and $\symbint$ represent an event
object and an integer, respectively.  Then, we can perform dynamic analysis
successfully until line 9.  On line 2, \jscode{length} is \jscode{2};
on line 4, \jscode{array} points to \jscode{this.pastEvents};
on lines 5--8, \jscode{args} stores an array with a single object stored in \jscode{e.events};
and on line 9, \jscode{isArray(array)} is \jscode{true}.
However, dynamic analysis fails for \jscode{copyArray(array)} on line 10
because the value of the \jscode{length} property of \jscode{array} is the {\sealed} value $\symbint$.
Then, we stop the {\sealed} execution, convert the current
{\sealed} state to its corresponding abstract state, and resume
the static analysis from line 10.  Because {\sealed} execution
leverages fast dynamic analysis as long as possible, the overall
analysis becomes more scalable.

\paragraph{Dynamic shortcuts for opaque functions.}
As the previous two examples additionally show, using dynamic shortcuts 
lessens the burden of modeling opaque functions from static analysis, and
it can even improve the analysis precision.
On line 9, since the \jscode{isArray} function is a JavaScript built-in library function,
it is implemented in a native language of the host environment, which
often requires manual modeling of its behaviors for JavaScript static analysis.
Assuming that a static analyzer models \jscode{isArray} to return the
boolean top value $\top_b$ that encompasses both \jscode{true} and \jscode{false},
static analysis of the ternary conditional expression on lines 9--10 analyzes
both branches \jscode{copyArray(array)} and \jscode{[array]}, even though
\jscode{[array]} is never reachable in the example code.  On the contrary,
using dynamic shortcuts, static analysis does not need to model \jscode{isArray}.
It can perform {\sealed} execution for \jscode{isArray}, which
returns a more precise result \jscode{true} than $\top_b$.
