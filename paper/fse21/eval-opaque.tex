\subsection{Opaque Function Coverage}
To evaluate how much manual modeling efforts of opaque functions
are reduced by dynamic shortcuts, we measured the number of tests for which
opaque functions are analyzed only by dynamic analysis not by
static analysis.  Table~\ref{table:func-replace} summarizes the result.
For 263 original tests and 167 abstracted tests that DS finished analysis, we measured the
number of tests that use only dynamic shortcuts instead of manual modeling
for each JavaScript built-in library function.  For each row,
\textbf{Object} column denotes a built-in object, \textbf{Function} a function
name, and \textbf{\#~Replaced} the number of tests successfully replacing manual
modeling via dynamic shortcuts over the total number of tests using the target function.
For example, the first row in the leftmost side describes that \jscode{Array} is used in
203 original tests and 116 abstracted tests.  Among them, 203 original
tests and 90 abstracted tests are successfully analyzed by using dynamic shortcuts
instead of manual modeling of \jscode{Array}.  Each filled cell describes
a fully replaceable case.  Therefore, dynamic
shortcuts effectively lessen the burden of manual modeling for JavaScript
built-in functions.  Again, since analysis of all the original Lodash 4 tests except one
finishes with a single dynamic shortcut, all the built-in functions
except \jscode{Math.floor} and \njscode{Function.prototype.call} are replaceable
for original tests.  For abstracted tests, 12 out of 60 built-in
functions are analyzed by only dynamic shortcuts.
